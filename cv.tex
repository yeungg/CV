%% -*- encoding: utf-8 -*-

\documentclass[11pt,a4paper,nolmodern]{moderncv}
\usepackage[noindent,UTF8]{ctex}
\usepackage{info}
\usepackage[colorlinks,linkcolor=red,anchorcolor=blue,citecolor=green]{hyperref}

\title{西北工业大学计算机学院}
\myquote{自强不息,厚德载物。}{}

\begin{document}
\setmainfont{Minion Pro}
\setsansfont{Myriad Pro}

\hyphenpenalty=10000
\maketitle

\section{教育背景}
\tlcventry{2006}{0}{\href{www.nwpu.edu.cn}{西北工业大学( NPU )},
  \href{http://jsj.nwpu.edu.cn/}{计算机学院},计算机科学工学博士}{}{}{}
\tlcventry{1999}{0}{\href{www.nwpu.edu.cn}{西北工业大学( NPU )},
  \href{http://jsj.nwpu.edu.cn/}{计算机学院},计算机科学与技术工学硕士}{}{}{}
\tlcventry{1994}{0}{中国人民解放军第二炮兵工程学院,自动控制系,计算机
  应用工学学士}{}{}{}
\begin{itemize}
  \item 大三学年排名:1/54,GPA:93.8/100
  \item 全部学期排名:3/54,GPA:90.8/100
  \item 2011年,清华大学综合优秀奖学金
  \item 2010年,清华大学综合优秀奖学金
\end{itemize}}

\section{能力}
\subsection{开发}
\cvcomputer{语言}{C/C++, C\#, Python, Java, JavaScript}
           {Web}{HTML, ASP .Net, jQuery, AJAX}
\cvcomputer{框架}{Django}
           {数据库}{MySQL, SQL Server, MongoDB, Redis}
\cvcomputer{代码管理}{SVN, Git}
           {工具}{Redmine, GitHub}

\subsection{其他}
\cvcomputer{Office}{iWork, OpenOffice/LibreOffice, Microsoft Office}
           {操作系统}{Mac OS X, GNU/Linux(Ubuntu, Mint), Windows}
\cvcomputer{排版}{\XeLaTeX{}}
           {编辑器}{Sublime Text,VIM}

\section{经历}
\subsection{实习经历}
\tldatelabelcventry{2012}{2012夏}{台湾国立清华大学IDEA实验室}{}{}{}{
  \begin{itemize}
    \item 研究n-gram模型在中文情感分析中的性能
  \end{itemize}
}
\vspace{0.5em}
\tldatelabelcventry{2012}{2012春}{清华大学软件学院信息系统与工程研究所}{}{}{}{
  \begin{itemize}
    \item 参与盘古搜索项目中OCR模块的代码修改工作>
  \end{itemize}
}

\subsection{项目经历}
\tlcventry{2011}{0}{人人网应用——清华晒课厅}{}{}{}{
\begin{itemize}
 \item 使用ASP .Net + SQL Server开发
 \item 应用安装量超过10000
\end{itemize}}
\vspace{0.5em}
\tlcventry{2009}{0}{其他课程项目、研究}{}{}{}{
\begin{itemize}
 \item 使用LDA和Collaborative的方法实现了文章推荐系统
 \item 使用Node.js、MongoDB和Redis实现了基于地理位置信息的移动应用程序
 \item 使用Lex和Yacc完成了简单Python程序到C\#的翻译器——Py\#(\href{https://github.com/terro/Py-Sharp}{GitHub链接})
 \item 在教学用操作系统xv6中使用FAT32替换了原有的简单文件系统
 \item 完成了基于Qt4的嵌入式MP3播放器的设计与实现(\href{https://github.com/terro/A-Simple-Embedded-MP3-Player-Based-On-ARM-9}{GitHub链接})
 \item 完成了具有加密、硬件信息测试功能的MBR开发
 \item 阅读了HyperSQL数据库源代码,并从存储、查询、事务、并发和恢复等方面对代码进行了分析
 \item 使用VHDL及C在Spartan-II 300E LC上完成了基于串口通信及LCD屏幕的猜拳游戏
 \item 使用TI MSP430单片机完成了基于超声波测距模块的距离和速度测量装置
\end{itemize}}

\subsection{社会工作}
\tlcventry{2011}{0}{清华大学软件学院学生科协副主席、主席}{}{}{}{
\begin{itemize}
 \item 主管软件学院科协的各项工作
\end{itemize}}
\vspace{0.5em}
\tlcventry{2011}{0}{清华大学勤工助学大队信息系统服务部部长}{}{}{}{
\begin{itemize}
 \item 协调部门内各项目组和队员的工作
\end{itemize}}
\vspace{0.5em}
\tlcventry{2010}{2011}{清华大学软件学院学生科协赛事部部长}{}{}{}{
\begin{itemize}
 \item 北京市高校软件设计邀请赛主办方负责人
 \item 组织、主办大一新生算法班
 \item 获清华大学学生科协优秀学生干部称号
\end{itemize}}

\subsection{公益活动}
\tlcventry{2009}{0}{首都机场志愿者、中国科技馆志愿者、中网志愿者}{}{}{}{
\begin{itemize}
 \item 累计完成志愿服务89小时
 \item 获清华大学三星志愿者称号
\end{itemize}}
\vspace{0.5em}
\tldatelabelcventry{2010}{2010夏}{中美联合支教赴青海乐都支教活动}{}{}{}{
\begin{itemize}
 \item 在青海乐都高级实验中学支教10天
\end{itemize}}

\section{语言}
\cvlanguage{汉语}{母语}{}
\cvlanguage{英语}{熟练}{通过全国大学英语四级考试,通过清华大学英语水平一考试}

\section{个人兴趣}

\cvhobby{运动}{游泳}
\cvhobby{互联网}{\href{https://twitter.com/terro1991}{Twitter}, \href{http://facebook.com/dangfan}{facebook}, \href{http://github.com/terro}{GitHub}, stackoverflow}
\cvhobby{其他}{旅游,读书,摄影}

\end{document}

