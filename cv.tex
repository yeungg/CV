%% -*- coding: utf-8 -*-

% arara: xelatex
% arara: biber
% arara: xelatex

\documentclass[unicode,11pt,a4paper,nolmodern]{moderncv}
%\usepackage[noindent,UTF8]{ctex}
\usepackage{xltxtra,fontspec,xunicode}
\usepackage[slantfont,boldfont]{xeCJK} % 允许斜体和粗体
\usepackage{info}

% \usepackage{zhnumber}
\setCJKmainfont{YouYuan}   % 设置缺省中文字体 徐静蕾字体
%\setCJKmonofont{SimSun}   % 设置等宽字体
%\setmainfont{Garamond} % 英文衬线字体
%\setmonofont{Consolas} % 英文等宽字体
%\setsansfont{Helvetica CE} % 英文无衬线字体
%\usepackage[backend=biber,style=caspervector, utf8,sorting=ydnt,seconds=true]{biblatex}
\usepackage[backend=biber,style=gb7714-2015,gbstyle=false,sorting=ydnt,seconds=true]{biblatex}
\usepackage{comment}
\usepackage{wasysym}
\usepackage{metre}
\usepackage{xcolor}
%\usepackage[colorlinks,linkcolor=red,anchorcolor=blue,citecolor=green]{hyperref}
\addbibresource{papers.bib}

\title{西北工业大学计算机学院}
\myquote{公诚勇毅,三实一新}{}
\newcommand\Colorhref[3][cyan]{\href{#2}{\small\color{#1}#3}}


\begin{document}
\hyphenpenalty=10000

\maketitle

\section{教育背景}
\tlcventry{2002}{2006}{计算机科学工学博士}{\href{www.nwpu.edu.cn}{西北工业大学(NPU)}}{\href{http://jsj.nwpu.edu.cn/}{计算机学院}}{}{}
  
\tlcventry{1999}{2002}{计算机科学与技术工学硕士}{\href{www.nwpu.edu.cn}{西北工业大学(NPU)}}{ \href{http://jsj.nwpu.edu.cn/}{计算机学院}}{}{}

\tlcventry{1994}{1998}{计算机应用工学学士}{中国人民解放军第二炮兵工程学院}{自动控制系}{}{}

\section{职业}
\tlcventry{2017}{2018}{工信部``高端装备人才''计划选拔赴德国交流``基于工业云的CPS与智能制造''}{}{}{}{}
\tlcventry{2012}{0}{\href{http://jsj.nwpu.edu.cn}{西北工业大学计算机学院}从事科研教学工作}{}{}{}{}
\tlcventry{2011}{2012}{\href{http://www.pdx.edu/}{Portland州立大学}和\href{http://http://berkeley.edu/}{UC Berkeley分校}访问学者}{}{}{}{}
\tlcventry{2006}{2011}{\href{http://jsj.nwpu.edu.cn}{西北工业大学计算机学院}从事科研教学工作}{}{}{}{}

\section{职务}
\tlcventry{2012}{0}{陕西省嵌入式系统技术重点实验室}{副主任}{}{}{}
\tlcventry{2019}{0}{陕西省计算机学会嵌入式系统专委会}{副主任}{}{}{}
\tlcventry{2019}{0}{中国计算机学会智能机器人专业组}{创始委员}{}{}{}
\tlcventry{2009}{0}{中国计算机学会协同计算专委会}{委员}{}{}{}
\tlcventry{2013}{0}{中国计算机协会}{高级会员}{}{}{}

\section{荣誉及获奖}
\tldatecventry{2023}{CCF科技成果奖”的技术发明二等奖}{信息物理融合系统一体化建模与定制化服务技术及应用}{\small {1}}{}{}
\tldatecventry{2010}{西北工业大学优秀青年教师}{}{}{}{}
\tldatecventry{2009}{西北工业大学``三育人''先进个人}{}{}{}{}
\tldatecventry{2009}{西北工业大学计算机学院研究生教学最满意教师}{}{}{}{}
\tldatecventry{2006}{国防科学技术进步二等奖}{\Asteriscus\,
  \Asteriscus\,\Asteriscus\,\Asteriscus\,\Asteriscus\,\Asteriscus 平台}{\small{2}}{}{}

\section{教学}
\tlcventry{2006}{0}{硕士研究生专业基础课}{分布计算系统原理}{}{}{}
\tlcventry{2016}{0}{本科专业基础课}{计算机操作系统}{}{}{}
\tlcventry{2017}{0}{本科专业基础课}{计算机组成原理}{\small\textbf{软件学院}}{}{}
\tlcventry{2012}{2015}{本科专业基础课}{计算机组成原理}{\small\textbf{计算机学院}}{}{}
\tlcventry{2007}{2012}{本科专业基础课}{汇编语言与接口技术}{}{}{}
\tlcventry{2007}{2009}{本科公共课}{微机原理}{}{}{}

\section{科研项目}
\subsection{正在承担的项目}

\tlcventry{2022}{2025}{自然基金重点项目}{泛在异构资源抽象与管理}{\small\textbf{主持}}{}{}
\tlcventry{2022}{2025}{军委科技委\Asteriscus\,\Asteriscus\,工程}{面向无\Asteriscus\,\Asteriscus\,\Asteriscus\,\Asteriscus\,\Asteriscus\,\Asteriscus\,研究}{\small\textbf{主持}}{}{}

\subsection{已完成的项目}
\tlcventry{2019}{2023}{JWKJW重大专项课题}{群体智能操作系统\Asteriscus\,\Asteriscus\,\Asteriscus\,\Asteriscus\,研究}{\small\textbf{主持}}{}{}
\tlcventry{2017}{2020}{国家重点研发计划课题}{智能无人系统异构资源管理
  与自主协同控制}{\small\textbf{主持}}{}{}
\tlcventry{2018}{2019}{``十三五''装备预研领域基金一般项目}{信息物理系统融合建模仿真方法}{\small\textbf{主持}}{}{}
\tlcventry{2019}{2020}{强基工程项目}{软件定义\Asteriscus\,\Asteriscus\,\Asteriscus\,\Asteriscus\,关键技术研究}{\small\textbf{主持}}{}{}
\tlcventry{2019}{2021}{军委科技委创新项目}{面向无人系统的群体智能评估方法}{\small\textbf{参加}}{}{}
\tlcventry{2016}{2017}{国防基础科研项目}{CPS软件\Asteriscus\,\Asteriscus\,\Asteriscus\,\Asteriscus\,开发技术}{\small\textbf{主持}}{}{}
\tlcventry{2015}{2016}{陕西省科技计划统筹项目}{应用可定制的CPS支撑平
  台}{\small\textbf{主持}}{}{}
\tlcventry{2015}{2018}{自然科学基金面上项目}{基于时空一致的CPS系统行为
  协同建模方法}{\small\textbf{主持}}{}{}
\tlcventry{2016}{2018}{智能制造新模式应用项目}{电子行业定制化智能制造建设项目}{\small\textbf{主持}}{}{}
\tlcventry{2011}{2015}{国防预研课题}{\Asteriscus\,
  \Asteriscus\,\Asteriscus\,\Asteriscus\,\Asteriscus 计算平台技术}{\small 参与/3}{}{}
\tlcventry{2013}{2014}{国家重大专项03专项子课题}{基站处理资源虚拟化关键技术研究}{\small\textbf{主持}}{}{}
\tlcventry{2010}{2012}{国家``核高基''重大专项子课题}{企业服务中间件设计与开发}{\small \textbf{主持}}{}{}
\tlcventry{2010}{2012}{国家``863''重点项目子课题}{可信国家软件资源共享与协同生产环境的大型软件应用研究}{\small\textbf{主持}}{}{}
\tlcventry{2005}{2010}{国防预研课题}{\Asteriscus\,\Asteriscus\,\Asteriscus\,\Asteriscus\,\Asteriscus\,\Asteriscus 保障技术}{\small 参与/2}{}{}
\tlcventry{2006}{2008}{``863''面上项目}{面向能力可信服务的自适应软件技术研究}{\small 参与/2}{}{}
\tlcventry{2006}{2007}{国家发改委产业化项目}{数据集成关键技术研究}{\small\textbf{合作主持}}{}{}

\section{科研方向}
%\todo[inline,caption={精炼研究方向}]{精炼研究方向,并给出一段描述。避免出现x与y这种方式}
\cvline{\textcolor{orange}{学科方向}}{计算机体系结构,计算机应用技术}
\cvline{\textcolor{orange}{研究方向}}{信息物理融合系统,群体智能无人操作系统,先进分布实时计算}
\cvline{\textcolor{orange}{简介}}{针对新型嵌入式系统结构网络化以及功能融合化的发展趋势,利
  用形式化的方法和工具,研究信息物理融合系统及嵌入式服务建模及分析技术,
  从而提升信息物理融合系统和嵌入式服务系统的服务质量与效能品质。相关研究成果已经在"嫦娥三号"等多个国防与民用重大工程得到应用。}

\nocite{*}
\printbibliography[title={已发表论文}]

\begin{comment}
\section{指导研究生}
\subsection{博士研究生}
\tldatecventry{2023}{学术研究生}{黄启明}{}{\small 指导}{}
\tldatecventry{2023}{学术研究生}{李明轩}{}{\small 协助}{}
\tldatecventry{2021}{学术研究生}{寇凯}{}{\small 指导}{}
\tldatecventry{2014}{学术研究生}{李梦洁}{}{\small 协助}{}
\tldatecventry{2021}{学术研究生}{李梦洁}{}{\small 协助}{}
\tldatecventry{2021}{学术研究生}{孙庆爽}{}{\small 协助}{}
\tldatecventry{2020}{学术研究生}{胡玉娇}{}{\small 协助, 已毕业}{} 
\tldatecventry{2022}{学术研究生}{武文亮}{}{\small 协助,已毕业}{}
\tldatecventry{2014}{学术研究生}{孙远}{}{\small 协助,已毕业}{}
\tldatecventry{2014}{学术研究生}{杜雪远}{}{\small 协助,已毕业}{}
%%\tldatecventry{2014}{学术研究生}{}{}{\small 协助,已毕业}{}
\tldatecventry{2013}{学术研究生}{沈博}{}{\small 协助,已毕业}{}
\tldatecventry{2012}{学术研究生}{孙中豪,杨亚磊}{}{\small 协助,已毕业}{}

\subsection{硕士研究生}
% TODO: 补充2021年和2022年的硕士研究生
%% 年度  类型  姓名 状态 论文题目
% TODO Complete the list of graduate students.
\tldatecventry{2021}{学术研究生}{王瑞哲,吴思凯,王岚清(女)}{\small 已毕业}{}{}{}{}
\tldatecventry{2021}{学术研究生}{王启,冯冬冬,谢嘉怡(女)}{\small 已毕业}{}{}{}{}
\tldatecventry{2020}{专业研究生}{何晓丽(女),翟开莉(女)}{\small 已毕业}{}{}{}{}
\tldatecventry{2019}{学术研究生}{张冬妮(女)}{\small 已毕业}{}{}{}{}
\tldatecventry{2018}{学术研究生}{袁艺文}{\small 已毕业}{}{}{}{}
\tldatecventry{2018}{专业研究生}{廉园园(女)}{\small 已毕业}{}{}{}{}
\tldatecventry{2017}{学术研究生}{杜庆龙,张鸽(女)}{\small 已毕业}{}{}{}{}
\tldatecventry{2016}{学术研究生}{杜三盛,张策}{\small 已毕业}{}{}{}{}
\tldatecventry{2015}{学术研究生}{王严,马雪超}{\small 已毕业(华为)}{}{}{}{}
\tldatecventry{2014}{学术研究生}{管涛}{\small 已毕业(华为)}{}{}{}{}

\section{常用工具}
\cvcomputer{编程语言}{$C/C^{++}$, Java, Python, Assembly}
                     {操作系统}{Windows, Linux(ubuntu),OSX}
\cvcomputer{代码管理}{Git,SVN}
                     {排版办公}{Microsoft Office, \LaTeX{},\XeLaTeX{}}

\section{语言}
\cvlanguage{汉语}{母语}{}
\cvlanguage{英语}{熟练}{通过``全国外语水平考试(WSK)''}

\section{个人兴趣}
\cvhobby{健身}{游泳}
\cvhobby{音乐}{上世纪80年代\textasciitilde{}90年代港台流行歌曲(粤语)}
\cvhobby{阅读}{中国古典文学,英语文学,历史}
\end{comment}
\end{document}

%%% Local Variables:
%%% mode: latex
%%% TeX-master: t
%%% End:
